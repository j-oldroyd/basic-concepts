\documentclass[addpoints,11pt]{exam}

% These are the packages I need.
\usepackage{amsfonts,amsmath,amssymb,amsthm,array,enumerate,mathtools,bm}

\usepackage[left=1in,right=1in,bottom=1in,top=.5in]{geometry}

% This is to create nicer tables.
\usepackage{booktabs,longtable}

% This apparently helps typesetting.
\usepackage{microtype}

% Provides useful macros, like \qty{}, \qty[], \qty(), etc.
\usepackage[italicdiff]{physics}

% Provides captions outside of figure environment and subfigures and FLOATS.
\usepackage{caption,subcaption,float}

\newtheorem{theorem}{Theorem}

\DeclarePairedDelimiter\innprod{\langle}{\rangle}
\DeclarePairedDelimiter\set{\{}{\}}
\DeclarePairedDelimiter\ceil{\lceil}{\rceil}
\DeclarePairedDelimiter\parens{\lparen}{\rparen}
\DeclarePairedDelimiter\brackets{[}{]}

\newcommand{\NN}{\mathbb{N}}
\newcommand{\ZZ}{\mathbb{Z}}
\newcommand{\RR}{\mathbb{R}}
\newcommand{\CC}{\mathbb{C}}
\newcommand{\HH}{\mathcal{H}}

% Here's where you edit the Class, Exam, Date, etc.
\newcommand{\class}{\textsc{Math 250-01}}
\newcommand{\term}{\textsc{Fall 2019}}
\newcommand{\examnum}{\textsc{Homework 1}}
\newcommand{\duedate}{\textsc{Due 9/11/18}}

% For an exam, single spacing is most appropriate
%\singlespacing
% \onehalfspacing
% \doublespacing

% For an exam, we generally want to turn off paragraph indentation
\parindent 0ex

% This places point total in the margin.
\pointsinmargin

% Changes addlinespace value
\setlength{\defaultaddspace}{.5cm}

% For listing table rows.
\newcounter{rowcount}
\setcounter{rowcount}{0}

\begin{document} 

%%%%%%%%%%%%%%%%%
% \todototoc
% \listoftodos
%%%%%%%%%%%%%%%%%
\pagebreak

% These commands set up the running header on the top of the exam pages
\pagestyle{headandfoot}
\header{\class}{\examnum\ - Page \thepage\ of \numpages}{\duedate}
% \footer{\oddeven{}{\pointbox}}{}{\oddeven{\pointbox}{}}
\headrule
\extraheadheight{.5cm}

\begin{coverpages}
\begin{flushright}
\begin{tabular}{p{2.8in} r l}
\class&&\hspace{3in} \\
\term&&\\
\examnum&&\\
\duedate&\textsc{Name}: & \makebox[2.83in]{\hrulefill}\\
\end{tabular}\\
\end{flushright}
\rule[1ex]{\textwidth}{.1pt}


This assignment contains \numquestions\ exercises. The following rules apply:
\begin{itemize}
\item \textbf{You must write clearly}. 
Your work must be organized in a reasonably neat and coherent way. 
Work scattered all over the page without a clear ordering will receive little to no credit.  

\item \textbf{You must show relevant work}.  
A correct answer, unsupported by calculations, explanation or algebraic work may not receive credit; an incorrect answer supported by substantially correct calculations and explanations might still receive significant partial credit. 

\item \textbf{You must staple this cover sheet to your homework} if you hand in a physical copy. 
Otherwise, I will not accept it.
If you choose to submit your homework electronically through Blackboard (\textit{in which case it must be turned in as a PDF}), then you are exempt from this requirement.
Either way, your assignment must be created using \LaTeX.

\item \textbf{Two problems will be graded meticulously}.
The remaining four problems will be graded in less detail and will be worth \textbf{four $\boldsymbol{(4)}$} points each, and two more points will be given freely to round out the total at 50 points. 
All grading will make use of the ``Rubric for Mathematical Writing'' which is available on Blackboard.
Your goal is to reach the level of ``Distinguished'' in all four categories.
Do not write in the following table.

\begin{center}
\textsc{\Large Grading Table}\\
\vspace{10pt}
\begin{tabular}{cccccc}
\toprule
Problem & Communication & Problem-solving & Computation & Understanding & Total  \\
\cmidrule(r){1-1}\cmidrule(lr){2-5}\cmidrule(l){6-6}
 & & & & & \fbox{\phantom{500} / 16}\\
\addlinespace[20pt]
 & & & & & \fbox{\phantom{500} / 16}\\
\addlinespace[20pt]
Remaining & & & & & \fbox{\phantom{500} / 16}\\
& & & & & $+2$ \\
\bottomrule
\addlinespace[10pt]
 & & & & & \fbox{\phantom{500} / 50} \\
\end{tabular}
\end{center}

\end{itemize}
\end{coverpages}

\newpage

\begin{questions}
\printanswers
\section*{Section 1.1}
  \question
  This problem is adapted from Problem 7 on page 1 of the text.
  Consider the following theorem:
  \begin{theorem}
    If $f$ is a quadratic function of the form $f(x) = ax^{2} + bx + c$ and $ac < 0$, then the function $f$ has two $x$-intercepts.
  \end{theorem}
  Using \textbf{only} this theorem, what can be concluded about the functions given by the following formulas?
  \begin{parts}
    \part $g(x) = -8x^{2} + 5x - 2$.

    \part $h(x) = -\frac{1}{3}x^{2} + 3x$.

    \part $k(x) = 8x^{2} - 5x - 7$.
  \end{parts}
  \begin{solution}
    \begin{parts}
      \part We can conclude that...

      \part We can conclude that...

      \part We can conclude that...
    \end{parts}
  \end{solution}

  \vspace{.5cm}

  \question
  Complete Problem 9 on page 14 of the text.

\section*{Section 1.2}
  \question
  Prove or disprove the following statement:
  \begin{quote}
    Let $a,b,c\in\CC$.
    Then there is only one value of $a$ such that $ax^{2} + bx + c = 0$ has exactly one solution.
  \end{quote}

  \vspace{.5cm}

  \question
  Let $m\in\ZZ$ be an integer that is precisely one more than a multiple of $4$.
  Is $m^{2}$ one more than a multiple of $8$?

  \vspace{.5cm}

  \question
  Complete Problem 10 on page 28 of the text.

  \vspace{.5cm}

  \question
  Complete Problem 13 on page 30 of the text.
\end{questions}
\end{document}